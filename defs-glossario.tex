
% Definições para glossario

% ATENCAO o SHARELATEX GERA O GLOSSARIO/LISTAS DE SIGLAS SOMENTE UMA VEZ
% CASO SEJA FEITA ALGUMA ALTERAÇÃO NA LISTA DE SIGLAS OU GLOSSARIO É NECESSARIO UTILIZAR A OPÇÃO :
% "Clear Cached Files" DISPONIVEL NA VISUALIZAÇÃO DOS LOGS 
% ---
% https://www.sharelatex.com/learn/Glossaries

% Normalmente somente as palavras referenciadas são impressas no glossário, portanto é necessário referenciar utilizando :
% \gls{identificação}            
% \Gls{identificação}            
% \glspl{identificação}            
% \Glspl{identificação}


% NOSSO GLOSSÁRIO

\newglossaryentry{Gamers} {
    name=gamers,
    description={Pessoa que joga jogos digitais}
}

\newglossaryentry{Entity} {
    name=Entity,
    description={Ferramenta de persistência de dados criada e mantida  pela Microsoft, com o objetivo de acelerar e auxiliar o desenvolvimento do código de acesso ao banco de dados}
}

\newglossaryentry{MySQL} {
    name=MySQL,
    description={Sistema de gerenciamento de banco de dados relacional (RDBMS) de código aberto que é amplamente usado para gerenciar e armazenar dados.}
}

\newglossaryentry{.NET} {
    name=.NET,
    description={Framework de desenvolvimento de software criado pela Microsoft. Fornece um ambiente de execução e uma biblioteca de classes abrangente para construir aplicativos modernos para diferentes plataformas, como Windows, web, dispositivos móveis e serviços em nuvem.}
}

\newglossaryentry{React} {
    name=React,
    description={Biblioteca JavaScript de código aberto, usada para criar interfaces de usuário (UI) interativas e reativas para aplicativos da web.}
}

\newglossaryentry{ReactNative} {
    name=React Native,
    description={Framework de desenvolvimento de aplicativos móveis criado pelo Facebook.}
}

\newglossaryentry{Csharp} {
    name=C\#,
    description={Linguagem de programação orientada a objetos criada e mantida pela Microsoft.}
}

\newglossaryentry{Javascript} {
    name=Javascript,
    description={Linguagem de programação para desenvolvimento de aplicações web.}
}

\newglossaryentry{Lua} {
    name=Lua,
    description={Linguagem de programação leve, de script, extensível e de propósito geral.}
}

\newglossaryentry{Sqlserver} {
    name=Sql Server,
    description={Sistema gerenciador de banco de dados relacional criado pela Sybase junto com a Microsoft.}
}

\newglossaryentry{Azure} {
    name=Azure,
    description={Plataforma de computação em nuvem oferecida pela Microsoft.}
}

\newglossaryentry{Back-end} {
    name=Back-end,
    description={Área da programação que se refere à parte do software que é responsável por processar dados, lógica de negócio, segurança e integrações.}
}

\newglossaryentry{Front-end} {
    name=Front-end,
    description={Parte do cliente de um aplicativo de software ou site com o qual os usuários interagem diretamente.}
}

\newglossaryentry{Kanban} {
    name=Kanban,
    description={Sistema visual de gerenciamento de fluxo de trabalho que visa aumentar a eficiência e a produtividade.}
}

\newglossaryentry{Jira} {
    name=Jira,
    description={Ferramenta de gerenciamento de projetos amplamente utilizada por equipes de desenvolvimento de software e gerenciamento de projetos.}
}

\newglossaryentry{Whatsapp} {
    name=Whatsapp,
    description={Aplicativo de mensagens instantâneas para dispositivos móveis e computadores.}
}

\newglossaryentry{Discord} {
    name=Discord,
    description={Plataforma de comunicação em tempo real projetada para comunidades online.}
}

\newglossaryentry{Twitch} {
    name=Twitch,
    description={Plataforma de transmissão de vídeo ao vivo.}
}

\newglossaryentry{Youtube} {
    name=Youtube,
    description={Plataforma de vídeos da Google.}
}

\newglossaryentry{Tiktok} {
    name=Tiktok,
    description={Plataforma de vídeos curtos, em apresentação majoritariamente vertical, voltada para dispositivos móveis.}
}

\newglossaryentry{Google} {
    name=Google,
    description={Mecanismo de busca.}
}

\newglossaryentry{Blog} {
    name=Blog,
    description={Plataforma online onde os usuários podem publicar conteúdo regularmente, como artigos, postagens, opiniões e outros tipos de materiais.}
}

\newglossaryentry{Microsoft} {
    name=Microsoft,
    description={Empresa de tecnologia.}
}

\newglossaryentry{Deploy} {
    name=deploy,
    description={Ato de implantar o site.}
}

\newglossaryentry{Vercel} {
    name=Vercel,
    description={Plataforma de nuvem para serviços de hospedagem de sites.}
}

\newglossaryentry{Rawg} {
    name=Rawg,
    description={Banco de dados de jogos digitais.}
}

\newglossaryentry{Identity} {
    name=Identity,
    description={Pacote para auxílio na autenticação e autorização do sistema.}
}

\newglossaryentry{Endpoints} {
    name=Endpoints,
    description={Pode ser considerado o ponto de extremidade de uma aplicação, na partes de protocolo de comunicação de um projeto servem para referenciar os pontos de comunicação de uma \ac{api}. Esses pontos servem para ligar os dados entre a aplicação do cliente e a aplicação do servidor.}
}

\newglossaryentry{Review} {
    name=Review,
    description={É um dos núcleos do projeto da GameLocker, consiste em uma avaliação que um usuário pode fazer sobre determinado jogo escolhido.}
}

\newglossaryentry{LandingPage} {
    name=Landing Page,
    description={Geralmente a página inicial de um site ou aplicação, esta página precisa conter todos os elementos e informações importantes do site e do negócio, com o objetivo de reter usuários para o serviço oferecido.}
}

\newglossaryentry{Framework} {
    name=Framework,
    description={Conjunto de ferramentas, bibliotecas e convenções que auxiliam os desenvolvedores a criar aplicativos ou sistemas de software.}
}

\newglossaryentry{Frameworks} {
    name=Frameworks,
    description={Conjunto de ferramentas, bibliotecas e convenções que auxiliam os desenvolvedores a criar aplicativos ou sistemas de software.}
}

\newglossaryentry{FullStack} {
    name=Full Stack,
    description={Abordagem de desenvolvimento capaz de lidar com todas as camadas de um aplicativo ou sistema, tanto o front-end quanto o back-end.}
}

\newglossaryentry{JavaSpring} {
    name=Java Spring Boot,
    description={Framework para desenvolvimento de APIs REST em Java, com bibliotecas robustas e suporte constante da Oracle}
}

\newglossaryentry{Jenkins} {
    name=Jenkins,
    description={Aplicação focada na automação de código de algum projeto, oferecendo diversos mecanismos e ferramentas para integração contínua, automação de testes, implantação de ambientes e entregas contínuas.}
}

\newglossaryentry{Xunit} {
    name=xUnit.net,
    description={Biblioteca de código aberta criada para realização de testes unitários focado para o ambiente de desenvolvimento .NET.}
}

\newglossaryentry{VisualStudio} {
    name=Visual Studio,
    description={Ambiente de desenvolvimento integrado criado e mantido pela Microsoft, principalmente com foco na criação de projetos nas linguagens C\# e Visual Basic.}
}

\newglossaryentry{CamelCase} {
    name=Camel Case,
    description={Convenção de nomenclatura na área da programação, que consiste em deixar a primeira letra de cada palavra em um termo composto em maiúsculo, com a única ficando em minúsculo sendo a primeira palavra.}
}

\newglossaryentry{PascalCase} {
    name=Pascal Case,
    description={Convenção de nomenclatura na área da programação, que consiste em deixar a primeira letra de cada palavra em um termo composto em maiúsculo.}
}

\newglossaryentry{Typescript} {
    name=Typescript,
    description={Linguagem de programação criada e mantida pela Microsoft, foi construída com base na linguagem de programação Javascript, mas com um foco maior em tipagem forte.}
}

\newglossaryentry{SOLID} {
    name=SOLID,
    description={Princípios no desenvolvimento de software no paradigma orientado a objetos.}
}

\newglossaryentry{CleanCode} {
    name=Clean Code,
    description={Conjunto de boas práticas no desenvolvimento de código e software, com foco no longo prazo e manutenibilidade de um sistema.}
}

\newglossaryentry{Angular} {
    name=Angular,
    description={Plataforma para desenvolvimento de aplicações web, foi criada pelo Google e seu funcionamento e objetivo é baseado na linguagem de programação Typescript.}
}

\newglossaryentry{Python} {
    name=Python,
    description={Linguagem de programação de alto nível, interpretada e de propósito geral. }
}

\newglossaryentry{ASPNET} {
    name=ASP.NET,
    description={Plataforma para desenvolvimento de aplicações web criada e mantida pela Microsoft, focada principalmente no uso da linguagem de programação C\#.}
}

\newglossaryentry{TechEmpower} {
    name=TechEmpower,
    description={Empresa de desenvolvimento de software e estudo de dados da área de tecnologia.}
}

\newglossaryentry{StackOverflow} {
    name=Stack Overflow,
    description={Site usado mundialmente e forte na área de tecnologia sobre perguntas e respostas envolvendo programação e áreas correlatas.}
}

\newglossaryentry{Facebook} {
    name=Facebook,
    description={Rede social amplamente usada pelo mundo.}
}

\newglossaryentry{Spotify} {
    name=Spotify,
    description={Plataforma de streaming de música e podcasts.}
}

\newglossaryentry{Xbox Live} {
    name=Xbox Live,
    description={Plataforma online da Microsoft destinado a proprietários de consoles Xbox.}
}

\newglossaryentry{Amazon Prime} {
    name=Amazon Prime,
    description={Serviço de assinatura premium oferecido pela Amazon.}
}

\newglossaryentry{Steam} {
    name=Steam,
    description={Plataforma digital de jogos digitais, com foco em venda de jogos e conexão de usuários.}
}

\newglossaryentry{AppService} {
    name=App Service,
    description={Plataforma de hospedagem gerenciada que permite aos desenvolvedores criar, implantar e dimensionar facilmente aplicativos da web e móveis.}
}

\newglossaryentry{AzureSQLDatabase} {
    name=Azure SQL Database,
    description={Serviço de banco de dados relacional totalmente gerenciado e baseado na nuvem da Microsoft Azure.}
}

\newglossaryentry{AzurePricingCalculator} {
    name=Azure Pricing Calculator ,
    description={Ferramenta fornecida pela Microsoft Azure que permite aos usuários estimar os custos dos serviços e recursos utilizados na plataforma Azure.}
}

\newglossaryentry{GoogleAdSense} {
    name=Google AdSense,
    description={Plataforma de publicidade online que permite aos proprietários de sites exibirem anúncios relevantes aos visitantes.}
}

\newglossaryentry{Multiplayer} {
    name=multiplayer,
    description={Interação entre vários jogadores em um videogame ou em um ambiente virtual.}
}

\newglossaryentry{Newzoo} {
    name=Newzoo,
    description={Empresa e site focada no estudo e análise de dados relacionado ao mercado global de jogos digitais.}
}

\newglossaryentry{SteamDB} {
    name=SteamDB,
    description={Site focado em distribuir a apresentar dados relacionada a plataforma de Jogos \gls{Steam}.}
}

\newglossaryentry{GrandViewResearch} {
    name=Grand View Research,
    description={Empresa de pesquisa de mercado e consultoria.}
}

\newglossaryentry{DataReportal} {
    name=DataReportal,
    description={Empresa de pesquisa global relacionada a internet e tecnologia.}
}

\newglossaryentry{Alvanista} {
    name=Alvanista,
    description={Rede social focada em jogos.}
}

\newglossaryentry{GGApp} {
    name=GG App,
    description={Site focado na descoberta de jogos e jogadores.}
}

\newglossaryentry{Origin} {
    name=Origin,
    description={Plataforma de distribuição digital.}
}

\newglossaryentry{Battle-Net} {
    name=Battle-Net,
    description={Serviço online de jogos eletrônicos.}
}

\newglossaryentry{AAA} {
    name=AAA,
    description={Jogos de alta qualidade e produção.}
}

\newglossaryentry{Indie} {
    name=indie,
    description={Jogos desenvolvidos por estúdios de menor porte.}
}

\newglossaryentry{Swagger} {
    name=Swagger,
    description={Ferramenta de código aberto amplamente utilizada para a documentação de APIs.}
}

\newglossaryentry{Valve} {
    name=Valve,
    description={Empresa norte-americana responsável por distribuir jogos digitais.}
}

\newglossaryentry{Blizzard} {
    name=Blizzard,
    description={Editora e desenvolvedora de jogos digitais.}
}

\newglossaryentry{MercadoPago}{
    name=Mercado Pago,
    description={
        Plataforma de pagamentos do Mercado Livre, criada para facilitar as transações financeiras entre lojistas e consumidores
    }
}

\newglossaryentry{Headers}{
    name=Headers,
    description={
        Cabeçalhos de requisições http, que passa informações adicionais, alterando ou melhorando a precisão da semântica da mensagem ou do corpo.
    }
}
