\section{Organização da equipe}

A equipe de projeto GameLocker, é formada por integrantes do curso de Análise e Desenvolvimento de Sistemas do \ac{ifsp}. A seguir uma breve descrição do perfil de cada integrante:

\begin{itemize}

\item \textbf{Alisson Kauan da Silva Santos}: Atualmente, ocupa a posição de estagiário de desenvolvimento \textit{\gls{Back-end}} na Unisys: Soluções de Tecnologia. Principal responsabilidade, trabalhar com o \textit{\gls{Framework}} \textit{\gls{.NET}} para a linguagem de programação \textit{\gls{Csharp}}, além de lidar com o banco de dados \textit{\gls{Sqlserver}}. Interesse e entusiasmo pelo desenvolvimento de sistemas;

\item \textbf{Brayan Yukio Uehara}: Atualmente, ocupa a função de Desenvolvedor de Software \textit{\gls{FullStack}} na empresa SATO Brasil, utiliza-se um conjunto diversificado de tecnologias para criar soluções eficazes. Principais áreas de atuação incluem o desenvolvimento de aplicativos e sistemas utilizando as linguagens de programação \textit{\gls{Csharp}}, \textit{\gls{Lua}} e \textit{\gls{Python}};

\item \textbf{Iuri Nicolas Henrique Garcia}: Atualmente, ocupa a função de estagiário em desenvolvimento de software \textit{\gls{FullStack}} na empresa itBeta², utilizando ferramentas como \textit{\gls{JavaSpring}}, \textit{\gls{MySQL}} e \textit{\gls{Angular}} para o desenvolvimento de aplicações Web. Auxilia também nas decisões sobre interfaces visuais nos projetos. Possui interesse maior por desenvolvimento \textit{\gls{Front-end}};

\item \textbf{Lucas de Lima Passos}: Atualmente, desempenha a função de Desenvolvedor Web e Mobile Júnior no Grupo GCB. Dentro dessa posição, se concentra no desenvolvimento de aplicações web utilizando a biblioteca \textit{\gls{React}}, bem como no desenvolvimento de aplicativos mobile usando o \textit{\gls{Framework}} \textit{\gls{ReactNative}}. Possui interesse particular no desenvolvimento mobile;

\item \textbf{Pedro Henrique Farias Boscachi}: Atualmente, ocupa a posição de estagiário de desenvolvimento \textit{\gls{FullStack}} na empresa BtCréditos. Trabalha com o \textit{\gls{Framework}} \textit{\gls{.NET}} para a linguagem de programação \textit{\gls{Csharp}} e com o banco de dados \textit{\gls{Sqlserver}}. Possui interesse significativo na área de desenvolvimento de sistemas.

\item \textbf{Rodrigo Shimizu Passos}: Atualmente, atua como estagiário em desenvolvimento de software \textit{\gls{FullStack}} na empresa itBeta², trabalhando com ferramentas como \textit{\gls{JavaSpring}}, \textit{\gls{MySQL}} e \textit{\gls{Angular}} no desenvolvimento de aplicações Web. Por vezes, auxilia na descoberta de requisitos nos projetos. Possui interesse particular por desenvolvimento \textit{\gls{Back-end}};

\end{itemize}

No contexto do projeto, as tarefas são divididas e atribuídas levando em consideração as disponibilidades de recursos e as habilidades dos integrantes da equipe, conforme o Quadro \ref{quadroOrganizacaoEquipe}.

\begin{quadro}[thb]
\centering
\ABNTEXfontereduzida
\caption{Organização da equipe}

\begin{tabular}{|l|c|c|c|c|c|c|}
\hline
\thead{Integrantes} & \thead{Alisson} & 
\thead{Brayan} & \thead{Iuri} & \thead{Lucas} & \thead{Pedro} & \thead{Rodrigo}\\
\hline
Back-end &  & X &  &  & X & X \
\\\hline
Blog &  & X &  &  & X & X \
\\\hline
Front-end & X &  & X & X &  &  \
\\\hline
Gestão & X & X & X & X & X & X  \
\\\hline
Documentação &  & X &  &  &  & X  \
\\\hline
\end{tabular}
\label{quadroOrganizacaoEquipe}
\fonte{Autores}
\end{quadro}

A equipe responsável pelo projeto colabora em conjunto para garantir a qualidade e eficiência de todas as etapas. Brayan, Pedro e Rodrigo assumem a responsabilidade pelo desenvolvimento do \textit{\gls{Back-end}}, onde se realiza a criação da \ac{api}, e na integração do banco de dados com a plataforma de nuvem \textit{\gls{Azure}}. Alisson, Lucas e Iuri são encarregados da programação do  \textit{\gls{Front-end}} e pela \ac{ux}/\ac{ui} do projeto.

No que tange à manutenção e atualização constante do Blog, Brayan, Pedro e Rodrigo assumem a importante missão de garantir que o conteúdo seja regularmente revisado e atualizado, garantindo a transparência e coerência das informações.

Em relação à documentação integral do projeto, todos os integrantes contribuem, todavia, Brayan e Rodrigo carregam a responsabilidade preponderante na adequação da documentação, garantindo que ela se alinhe com os padrões estabelecidos. 

A gestão deve ser realizada por todos os membros, garantindo que estejam alinhados com suas respectivas tarefas e contribuam no registro das atividades e dos processos de desenvolvimento.