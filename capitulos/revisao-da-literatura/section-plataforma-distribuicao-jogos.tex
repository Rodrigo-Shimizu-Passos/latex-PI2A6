\section{Plataformas de Distribuição de Jogos}

As Plataformas de Distribuição de Jogos representam os ambientes virtuais onde jogadores podem adquirir seus jogos de maneira digital. Uma pesquisa conduzida pela empresa \cite{games_distribution_platform}, especializada em análise de dados e tecnologia com sede no Reino Unido, aponta que as principais plataformas de distribuição são a \textit{\gls{Steam}} \textit{(\gls{Valve})}, a \textit{\gls{Origin}} \textit{(\ac{ea})} e a \textit{\gls{Battle-Net}} \textit{(\gls{Blizzard})}).

Algumas destas plataformas vão além da mera distribuição e oferecem funcionalidades adicionais, como gerenciamento de bibliotecas e integração de serviços. Um exemplo é a \textit{Epic Games Store}, que introduziu o \textit{Epic Gaming Manager}, um aplicativo de gerenciamento que proporciona aos jogadores uma experiência centralizada para acessar e atualizar seus jogos disponíveis na loja. Ademais, a plataforma \textit{Epic Online Services}, uma suíte de ferramentas e serviços que capacita os desenvolvedores de jogos a criar, lançar e gerenciar experiências interativas online \textit{multiplayer} e outros tipos de jogos. A plataforma oferece uma variedade de recursos, incluindo gerenciamento de identidade e contas, com soluções seguras para autenticação de jogadores e gestão de perfis de jogador. Além disso, conta com um sofisticado sistema de \textit{matchmaking} que permite aos jogadores encontrar partidas com outros jogadores que possuem habilidades e preferências semelhantes. 

\subsection{Steam}

A \textit{\gls{Steam}}, lançada pela \textit{\gls{Valve} Corporation} em 2003, se coloca como uma plataforma de jogos online de destaque. Ela emerge como uma das maiores plataformas de distribuição digital de jogos para PC a nível global, conferindo aos usuários uma extensa gama de funcionalidades. Através da \textit{\gls{Steam}}, os jogadores podem adquirir e fazer \textit{download} direto dos jogos em seus computadores, e recebem atualizações automáticas sempre que novas versões são lançadas. A plataforma também agrega elementos sociais, incluindo a possibilidade de adicionar amigos, integrar-se a grupos e participar de bate-papos, fomentando a interconexão entre jogadores ao redor do mundo.

A grande diversidade de títulos é uma das proeminentes vantagens da \textit{\gls{Steam}}, que oferece uma biblioteca com mais de 30.000 jogos disponíveis para \textit{download}, abarcando tanto títulos \textit{\gls{AAA}} quanto independentes. Somado a isso, a \textit{\gls{Steam}} regularmente oferece promoções e descontos em jogos, democratizando o acesso aos jogadores de todas as vertentes.

\subsection{Origin}

A \textit{\gls{Origin}}, desenvolvida pela \ac{ea}, uma das maiores empresas globais de entretenimento interativo, proporciona aos jogadores uma plataforma de distribuição digital de jogos. O intuito da \textit{\gls{Origin}} é conferir aos jogadores uma experiência acessível e cômoda para aquisição e jogo de seus jogos preferidos em seus computadores.

Através da \textit{\gls{Origin}}, os usuários obtêm acesso a uma variada gama de jogos, incorporando títulos \textit{\gls{AAA}} e independentes, além de DLCs, demonstrações e jogos gratuitos. Uma característica distintiva da plataforma é o seu \textit{feed} social, semelhante a redes sociais, permitindo que os jogadores sigam amigos e compartilhem suas conquistas nos jogos. Além disso, a \textit{\gls{Origin}} oferece notificações de descontos e oportunidades para experimentar versões beta exclusivas de jogos antes do lançamento oficial. A plataforma garante ainda atualizações automáticas para assegurar que os jogos estejam sempre atualizados e funcionando devidamente.

\subsection{Battle-Net}

Desenvolvida pela \textit{\gls{Blizzard} Entertainment}, uma das principais empresas do mundo dos jogos, a \textit{\gls{Battle-Net}} proporciona aos jogadores uma plataforma que almeja facilitar a jogabilidade de seus jogos mais renomados, incluindo títulos como \textit{World of Warcraft}, \textit{Diablo}, \textit{Overwatch} e \textit{Starcraft}.

Através da \textit{\gls{Battle-Net}}, os jogadores se deparam com uma série de recursos, englobando jogos, fóruns, lojas online e serviços de suporte. As opções de jogo são variadas, permitindo partidas online com outros jogadores ou jogar individualmente. Um fator notável da plataforma é o foco na segurança, com um sistema de autenticação sólido que resguarda as contas dos jogadores contra atividades fraudulentas.