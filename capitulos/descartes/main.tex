\chapter{Descartes/Escolhas}

Este capítulo busca descrever as mudanças relacionadas às regras de negócio e funcionalidades do projeto, divididas em descartes e escolhas. Além disso, são citadas algumas implementações futuras, que podem ser incluídas no projeto mesmo após sua conclusão.

\section{Descartes}

Uma funcionalidade que foi deliberadamente descartada reside na capacidade de conectar deliberadamente as contas e os perfis dos usuários com suas conquistas e progressos em outras plataformas, exemplificadas pela \textit{\gls{Steam}} e \textit{\gls{Microsoft}}. A incorporação dessa funcionalidade implicaria em um grau considerável de complexidade, agravado pelo desafio de obter acesso à \ac{api} da \textit{\gls{Steam}}, uma circunstância que poderia acarretar complicações suplementares no decorrer do processo de desenvolvimento do projeto. Portanto, essa funcionalidade foi excluída da abordagem, a fim de garantir um escopo mais gerenciável e alinhado com os objetivos principais estabelecidos.

A implementação do anúncio \textit{\gls{GoogleAdSense}} tornou-se inviável devido a problemas existentes, notadamente a insuficiência de tráfego no site. A ausência de tráfego relevante é um problema complexo, envolvendo questões como otimização de mecanismos de busca (SEO), qualidade do conteúdo e estratégias de marketing digital. Nesse cenário, tornou-se claro que a implementação do \textit{AdSense} não era viável no momento presente. O baixo tráfego do site comprometeria a eficácia dos anúncios, levando a baixas taxas de visualizações e, por conseguinte, a receitas insignificantes.

Outra funcionalidade adicional que foi descartada relaciona-se com a disponibilização de notícias relevantes no contexto dos jogos eletrônicos. Isso abrangeria aspectos como a repercussão de novos lançamentos e eventos relacionados ao universo dos jogos.

\section{Escolhas}

A escolha de se implementar um sistema \textit{Web} está diretamente relacionada ao fato de que a grande maioria dos jogos está disponível para PC, além da forte presença da \textit{\gls{Steam}}, que comumente é acessada em computadores. Por isso, espera-se que a maioria dos usuários da plataforma GameLocker a acessem em navegadores \textit{Web}.%INCLUIR FONTE

Outra decisão a se apontar foi a de não armazenar o histórico de mensagens da arena de jogos, que se dá por dois fatores principais: recursos computacionais e complexidade de desenvolvimento. O primeiro se justifica pelo fato de que manter armazenar um número muito elevado de mensagens da arena poderia ser muito custoso computacionalmente. O segundo tem relação com o cronograma do projeto, já que desenvolver uma maneira de armazenar essas mensagens poderia atrasar os prazos definidos.

\section{Implementações Futuras}

Como implementação futura, é possível citar a publicação de jogos diretamente na plataforma, permitindo que pequenos desenvolvedores de jogos conseguissem um espaço para divulgá-los livremente. Esta funcionalidade foi sugerida pelo professor orientador do primeiro semestre da disciplina, Carlos Henrique Veríssimo, como uma maneira de propagar os jogos desenvolvidos por alunos de cursos técnicos.

Apesar do \textit{\gls{GoogleAdSense}} ter sido descartado, a equipe tem o planejamento de implementar essa estratégia de anúncio assim que o tráfego da aplicação melhorar. A implementação do \textit{\gls{GoogleAdSense}}, foram considerados fatores cruciais, como o perfil do público-alvo da aplicação, de modo a aproveitar as oportunidades de monetização e direcionar anúncios relevantes aos visitantes da plataforma, proporcionando uma experiência agradável e maximizando o potencial de geração de receita. Foi estabelecido adotar o modelo de remuneração \ac{cpm} como forma de monetização pelo número de vezes que os anúncios seriam exibidos na plataforma, a cada mil impressões.

Outra implementação futura seria a disponibilização de uma versão \textit{mobile} da aplicação, para assim atingir uma maior gama de usuários e otimizando a usabilidade do sistema.