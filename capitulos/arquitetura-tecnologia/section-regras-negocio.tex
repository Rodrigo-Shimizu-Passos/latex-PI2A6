\subsubsection{Regras de Negócio}
\label{sec:regras_negocio}

Os Quadros \ref{tab:regrasdenegocio1}, \ref{tab:regrasdenegocio2} e \ref{tab:regrasdenegocio3} estão organizados por temas, com o propósito de aprimorar a organização e a compreensão do conteúdo. Estes quadros têm como objetivo descrever de maneira abrangente todas as regras de negócio propostas, as quais devem ser integralmente incorporadas nos requisitos funcionais e não funcionais do projeto.

\begin{quadro}[h!]
\centering
\caption{Regras de Negócio de Login, Autenticação e Usuário}
\label{tab:regrasdenegocio1}
\begin{longtable}{|p{2.5cm}|p{10.0cm}|p{2.5cm}|}
\hline
ID & Descrição & Relacionados
\\\hline
RN1 & Ao acessar o site, para ter acesso aos conteúdos e funcionalidades, o usuário deve estar previamente autenticado &  \
\\\hline
RN2 & O acesso do usuário ao sistema de login é restrito àqueles que possuem um cadastro ativo &  \
\\\hline
RN3 & Durante o processo de cadastro, é imprescindível que o usuário confirme seu endereço de e-mail a fim de validar seu registro &  \
\\\hline
RN4 & Deve ser enviado um e-mail para redefinição de senhas de usuários &  \
\\\hline
RN5 & Cada usuário deve possuir um perfil individual, através do qual poderá administrar seus dados pessoais e informações relacionadas aos jogos &  \
\\\hline
RN6 & O usuário possui a prerrogativa de editar seus próprios dados pessoais &  \
\\\hline
RN7 & É facultado ao usuário o direito de excluir sua conta a qualquer momento &  \
\\\hline
\end{longtable}
\fonte{Os Autores.}
\end{quadro}

\begin{quadro}[h!]
\centering
\caption{Regras de Negócio de Gerenciamento de Jogos e Assinatura Premium}
\label{tab:regrasdenegocio2}
\begin{longtable}{|p{2.5cm}|p{10.0cm}|p{2.5cm}|}
\hline
ID & Descrição & Relacionados
\\\hline
RN8 & Ao selecionar um jogo, o usuário tem a possibilidade de adicioná-lo ao seu perfil &  \
\\\hline
RN9 & No momento de adicionar o jogo, é exigido que o usuário forneça informações básicas sobre o jogo &  \
\\\hline
RN10 & Após a inclusão, o usuário terá permissão para editar as informações dos seus jogos &  \
\\\hline
RN11 & No perfil do usuário, devem estar disponíveis todas as listagens de jogos, organizadas com base em seu status &  \
\\\hline
RN12 & Os usuários devem ter a capacidade de visualizar todos os jogos e avaliações nos perfis de outros usuários &  \
\\\hline
RN13 & O usuário terá a opção de filtrar os jogos com base em determinados parâmetros &  \
\\\hline
RN14 & O usuário deve ser capaz de avaliar um jogo, atribuindo uma nota e fornecendo uma crítica &  \
\\\hline
RN15 & O usuário deve ter a capacidade de excluir os jogos que adicionou a qualquer momento &  \
\\\hline
RN16 & Os usuários serão categorizados em usuários comuns e usuários premium &  \
\\\hline
RN17 & O sistema deve incorporar um modelo de assinatura premium, onde os usuários que pagarem uma quantia específica terão acesso a determinados benefícios e funcionalidades exclusivas &  \
\\\hline
RN18 & Usuários comuns não terão acesso a benefícios extras &  \
\\\hline
RN19 & Os usuários premium têm o direito de cancelar sua assinatura a qualquer momento &  \
\\\hline
\end{longtable}
\fonte{Os Autores.}
\end{quadro}


\begin{quadro}
\centering
\caption{Regras de Negócio da Arena de Jogos}
\label{tab:regrasdenegocio3}
\begin{longtable}{|p{2.5cm}|p{10.0cm}|p{2.5cm}|}
\hline
ID & Descrição & Relacionados
\\\hline
RN20 & O sistema deve oferecer uma aba dedicada às arenas &  \
\\\hline
RN21 & Cada arena deve possuir um administrador designado, sendo restrito aos usuários premium o direito de se tornarem administradores de arenas &  \
\\\hline
RN22 & Ao criar uma arena, o administrador é responsável por definir informações essenciais, como o jogo a ser jogado e a descrição da arena &  \
\\\hline
RN23 & O administrador tem a autorização para excluir sua arena a qualquer momento &  \
\\\hline
RN24 & Cada arena contará com um chat exclusivo, permitindo a comunicação entre todos os jogadores presentes na arena &  \
\\\hline
\end{longtable}
\fonte{Os Autores.}
\end{quadro}