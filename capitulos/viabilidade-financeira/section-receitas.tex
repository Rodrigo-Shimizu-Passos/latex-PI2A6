\section{Receitas}

Visando assegurar a viabilidade financeira do projeto, foram definidas estratégias de remuneração através de anúncios de banners personalizados. Além disso, foi estabelecido um plano Premium que oferece recursos e benefícios adicionais em comparação com o plano gratuito.

\subsection{Anúncios de banners personalizados}

A implementação de anúncios por meio de banners personalizados surgiu em resposta a dificuldades enfrentadas durante a implementação do \textit{\gls{GoogleAdSense}}. Esta estratégia será empregada até que os problemas relacionados ao baixo tráfego sejam resolvidos. Apesar das adversidades encontradas, os anúncios personalizados oferecem a vantagem de adaptar sua aparência e estilo para se integrar de forma harmoniosa com o tema visual da aplicação. Além disso, proporcionam a capacidade de personalizar os tipos de anúncios exibidos, assegurando assim que o conteúdo publicitário seja relevante para o público-alvo.

\subsection{Plano Premium}

O plano Premium consiste em uma assinatura oferecida aos usuários, proporcionando benefícios exclusivos, tais como bloqueio de anúncios e criações de Arenas de Jogos. Conforme descrito abaixo:

\begin{itemize}
    \item \textbf{Bloqueio de anúncios}: Experiência livre de anúncios, podendo navegar no site sem interrupções de propagandas indesejadas;
     \item \textbf{Criações de Arenas de Jogos}: Inclusão de uma arena de jogos online, com o intuito de proporcionar aos usuários a oportunidade de competir entre si em jogos \textit{\gls{Multiplayer}}, além de criar torneios e interagir com outros jogadores;
\end{itemize}

As assinaturas podem ser efetuadas de forma mensal, permitindo que os usuários cancelem ou alterem sua assinatura a qualquer momento, sem compromisso a longo prazo. Ou anual, adequada para aqueles que desejam se comprometer com o serviço ou produto a longo prazo e estão dispostos a fazer um pagamento adiantado. Opções de planos premium disponíveis descritos abaixo:

\begin{itemize}
    \item \textbf{Assinatura mensal}: R\$ 15,00;
    \item \textbf{Assinatura anual}: R\$ 160,00
\end{itemize}

É perceptível que a opção pela subscrição anual proporciona aos usuários um desconto de aproximadamente 11\% quando comparada à alternativa mensal. Esta estratégia visa não somente otimizar o custo para o cliente, mas também objetiva consolidar uma relação de fidelização ao estender o compromisso de uso ao longo do ano, contribuindo, assim, para a captação de recursos sustentáveis.

O valor mensal estipulado em R\$ 15,00 é estabelecido com o propósito de subsidiar as despesas inerentes à plataforma, ao passo que visa também a assegurar proventos futuros. Destarte, esse montante é calculado de maneira a não onerar excessivamente o usuário final, ao mesmo tempo em que se almeja construir um cenário de lucratividade sustentável para a empreitada.

Com a finalidade de estabelecer esse valor mensal, tomou-se como base a análise realizada no tópico \ref{subsec::publico-alvo_semelhante}, já que as aplicações analisadas oferecem recursos similares dentro dessas subscrições, a exemplo de personalização de perfis, bloqueio de anúncios e acesso a funcionalidades exclusivas, além de possuírem um público-alvo parecido com o da GameLocker, como já foi sugerido.

\subsubsection{Análise das Assinaturas}

Após minuciosa análise dos valores adotados por ambas as plataformas e das vantagens inerentes a cada uma, somada a uma avaliação justa dos ganhos projetados e dos custos operacionais do sistema, é possível deduzir que a tarifa mensal de R\$ 15,00 representa um montante equitativo a ser requisitado dos consumidores. Este valor permanece em consonância com as tarifas observadas em sistemas similares, ao mesmo tempo que se esforça por manter um nível de acessibilidade abrangente, abarcando assim uma considerável parcela de potenciais usuários.