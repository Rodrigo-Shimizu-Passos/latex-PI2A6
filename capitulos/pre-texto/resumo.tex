\setlength{\absparsep}{18pt}
\begin{resumo} 

A indústria dos jogos digitais emerge como um protagonista poderoso no cenário global, ultrapassando até mesmo o setor cinematográfico em termos de impacto e alcance. Em 2021, enquanto o mercado cinematográfico registrou uma receita de US\$ 90,9 bilhões de dólares \cite{global_films_musics_market}, a indústria global de jogos atingiu a notável marca de US\$ 192,7 bilhões de dólares no mesmo período \cite{global_games_market}. Esse crescimento exponencial evidencia o apelo massivo dos jogos, representando uma mudança cultural significativa, na qual os consumidores dedicam uma parte substancial de seu tempo a essa forma de entretenimento interativo. Nesse contexto dinâmico, a \textit{GameLocker} surge como resposta à crescente demanda por experiências de jogo mais organizadas e socialmente envolventes. Ao compreender a necessidade de fornecer um local centralizado para o armazenamento e organização de bibliotecas de jogos, bem como um espaço interativo e amigável, a \textit{GameLocker} se destaca como uma inovação crucial. Além de simplificar a gestão de jogos, a plataforma visa criar uma comunidade global coesa, onde jogadores de diferentes origens geográficas podem se conectar e compartilhar suas paixões comuns. Ao fomentar a interação social em um ambiente digital, a \textit{GameLocker} não apenas facilita competições amistosas, mas também promove uma troca rica de experiências entre os jogadores. Ao proporcionar uma solução prática e, ao mesmo tempo, cultivar um ambiente que nutre a paixão e o engajamento dos jogadores, a \textit{GameLocker} se posiciona como uma iniciativa integralmente alinhada com as necessidades e desejos da crescente comunidade global de entusiastas dos jogos.

    \vspace{\onelineskip}

    \textbf{Palavras-chave:} indústria dos jogos digitais, impacto e alcance, gestão de jogos, interação social, entusiastas dos jogos.
    
\end{resumo}

\pagebreak